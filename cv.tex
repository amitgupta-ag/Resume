\documentclass{tccv}
\usepackage[english]{babel}

\begin{document}

\part{Amit Gupta}

\section{Work experience}

\begin{eventlist}

\item{September 2017 -- Present}
     {Scania CV AB, Sweden}
     {Systems Architect}\newline
        Responsible for security architecture for vehicle on-board electrical systems. Responsibilities include designing secure architecture, security risk assessment, security tools prototyping, assisting development teams with standard practices.\newline
        :: Key Projects include - Public key and security certificate infrastructures (PKI), V2X communication standards, Electric vehicles.

\item{February 2017 -- September 2017}
     {Scania CV AB, Sweden}
     {Master Thesis - cybersecurity}

        The Master thesis proposes an original security risk assessment methodology for Ethernet-based vehicular networks. The methodology is a four-step iterative process model and offers a structured approach to design, analyze, assess, mitigate and record rationale behind assessments.\newline
        :: \href{http://essay.utwente.nl/73894/}{Publication}
        
\item{July 2012 -- September 2015}
     {GE Healthcare, India and USA}
     {Senior Software Developer} \newline Worked as full-stack developer on multiple global projects together with highly diverse global and fast-paced teams. 
     \newline :: Diagnostics cardiology - Designed and developed front-end for premium segment ECG recording devices.
     \newline 
    \begin{yearlist}
    \item{::}{Diagnostics Cardiology}{Designed and developed front-end for premium segment ECG recording devices}  
    \item{::}{Software Platforms}{Implemented web-services for storage of DICOM images and performing search in encrypted domain}
\end{yearlist}

%\section{Public projects}
\section{Certifications}
\begin{yearlist}
    \item{::}{Udemy}{Hands on security penetration testing using Kali Linux }  
    \item{::}{Asian School of Cyberlaws}{International Cyberlaws}
    \item{::}{International Association for Six Sigma Certification (IASSC)}{Trained for lean Six Sigma Green Belt}
    \item{::}{General Electric Co., USA}{Edison Engineering Development Program (EEDP)}
    %\item{::}{General Electric Co., USA}{Foundations of business leadership}
\end{yearlist}

% \begin{itemize}
% \item\textbf{one}{Penetration testing using Kali Linux : Udemy}
% \item\textbf{one}{International Cyberlaws : Asian School of Cyber Laws (ASCL)}
% \end{itemize}
\end{eventlist}

% \section{Certifications}
%   \begin{itemize}
%       \item One
%       \item two
%   \end{itemize} 
%     {Penetration testing using Kali Linux : Udemy}
%     {International Cyberlaws : Asian School of Cyber Laws (ASCL)}
%     {Foundations of business leadership : General Electric Co. USA}
% \end{eventlist}

\personal
    [http://www.amitgupta.eu]
    {lgh 1203, Storgatan 26,Sodertalje -- 151 36}
    {+46 (0) 765 167 404}
    {amitgupta.eu@gmail.com}

\section{Education}

\begin{yearlist}

\item[Grade: 8/10]{2015 -- 2017}
     {MS in Cybersecurity \& Privacy}
     {University of Twente, NL\newline University of Trento, IT}
     
\item[Grade: 7.82/10]{2008 -- 2012}
     {B.Tech in Computer Science}
     {National Institute of Technology - Bhopal, IN}

\end{yearlist}

\section{Public projects}
\begin{yearlist}
\item{::}{Evolo - A wireless drone hacking device to secure user’s privacy}{Built a security device using raspberryPi3, Arduino and wireless module, which could protect a user from unknown drone squinters that try to intrude into their private lands.\newline\href{https://www.youtube.com/watch?v=1HGsP9x9wTU}{YouTube} : \href{https://github.com/amitgupta46/Evolo}{GitHub}}

\item{::}{Code Analysis of Hacking Team’s Remote-Control System (RCS) exploit kit}{Performed Static Code Analysis of Hacking Team’s exploit kit, called RCS. The code analysis was awarded the best report in the batch.\newline\href{http://securitylab.disi.unitn.it/lib/exe/fetch.php?media=teaching\%3Aofftech\%3A2015\%3Areports\%3Acodeanalysis_hackingteam.pdf}{Published Paper}}
     
\item{::}{Two Fish block cipher encryption Algorithm}{Implemented high performance encryption and decryption for Twofish. Twofish is a 16-round Feistel network 128-bit block cipher that accepts a variable-length key up to 256 bits.\newline\href{https://github.com/amitgupta46/Magma}{GitHub}}

% \item{::}{Conference on operational research Euro2016, Poland}{Published my research, which suggested decision matrices to the football team managers and players for making optimal career choices.\newline\href{http://www.euro2016.poznan.pl/}{Euro2016 Poznan}}
\end{yearlist}

% \section{Communication skills}

% \begin{factlist}
% \item{English/Hindi}{Native}
% \item{Bengali}{Native speaker}
% \item{Spanish}{Oral: good}
% \end{factlist}

\section{Software skills}

\begin{factlist}

\item{Security}
     {Risk Assessment, Architecture, PKI, ETSI, GDPR, Penetration testing}

\item{Coding}
     {Javascript Frameworks, Python, C, scripts }

\item{Platforms}
     {Windows, Linux, Mac}

\end{factlist}

\end{document}
